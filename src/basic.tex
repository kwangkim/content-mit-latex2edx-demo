%%%%%%%%%%%%%%%%%%%%%%%%%%%%%%%%%%%%%%%%

\begin{edXchapter}{Basic examples}

%%%%%%%%%%%%%%%%%%%%%%%%%%%%%%%%%%%%%%%%%%%%%%%%%%%%%%%%%%%%%%%%%%%%%%%%%%%%%

\begin{edXsequential}{Basic example problems}

\begin{edXvertical}

\begin{edXproblem}{Option response}{weight="10"}

This is a sample problem, which is worth 10 points.

Give the correct python \texttt{type} for the following expressions.  Select \texttt{noneType} if the expression is illegal.

\begin{itemize}
\item \edXinline{\texttt{3}~~~}   \edXabox{expect="int" options="noneType","int","float" type="option" inline="1"}
\item \edXinline{\texttt{5.2}~~~} \edXabox{expect="float" options="noneType","int","float" type="option" inline="1"}
\item \edXinline{\texttt{3/2}~~~} \edXabox{expect="int" options="noneType","int","float" type="option" inline="1"}
\item \edXinline{\texttt{1+[]}~~~} \edXabox{expect="noneType" options="noneType","int","float" type="option" inline="1"}
\end{itemize}

\edXgitlink{\giturl}{Source TeX}

\end{edXproblem}

%%%%%%%%%%%%%%%%%%%%%%%%%%%%%%%%%%%%%%%%

\begin{edXproblem}{String response}

What state is Detroit in?

\edXabox{expect="Michigan" type="string"}

\begin{edXsolution}

Explanations can also be provided inside:
\begin{verbatim}
\begin{edXsolution}
... 
\end{edXsolution}
\end{verbatim}

\end{edXsolution}

\edXgitlink{\giturl}{Source TeX}
\end{edXproblem}


\end{edXvertical}

%%%%%%%%%%%%%%%%%%%%%%%%%%%%%%%%%%%%%%%%%%%%%%%%%%

\begin{edXvertical}


\begin{edXproblem}{Multiple choice single answer}{}

What color is the sky on a clear sunny day?

\edXabox{type="multichoice" 
  expect="Blue"
  options="Red","Green","Blue","Black" }

\edXgitlink{\giturl}{Source TeX}
\end{edXproblem}


\begin{edXproblem}{Multiple choice multiple answers}{}

What are the best computer programming language?  Choose {\em all} which apply:

\edXabox{type="oldmultichoice" 
  expect="Python","C++"
  options="Cobol","Pascal","Python","C++","Clu","Forth"
 }

\edXgitlink{\giturl}{Source TeX}
\end{edXproblem}

\end{edXvertical}

%%%%%%%%%%%%%%%%%%%%%%%%%%%%%%%%%%%%%%%%%%%%%%%%%%

\begin{edXvertical}

%%%%%%%%%%%%%%%%%%%%%%%%%%%%%%%%%%%%%%%%

\begin{edXproblem}{Numerical response}

\section*{Example of numerical response}  

What is the numerical value of $\pi$?

\edXabox{expect="3.14159" type="numerical" tolerance='0.01' }

\edXgitlink{\giturl}{Source TeX}
\end{edXproblem}


\begin{edXproblem}{Numerical response with inline labels}{}

Numerical response questions can have inline labels, both before and after the input box:

\begin{itemize}

\item  

    \edXinline{$^2P_{3/2}~~F=3$ $~~~~~ g= $ } \edXabox{expect="0.66666" type="numerical" tolerance='3\%' inline='1' }
    \edXinline{Hz}

\item  

    \edXinline{$^2S_{1/2}~~F=2$ $~~~~~ g= $ } \edXabox{expect="0.5" type="numerical" tolerance='3\%' inline='1' }
    \edXinline{Joules/sec}

\end{itemize}

\begin{edXsolution}

For the stretched states the formula is unnecessary: all the angular momenta are
then aligned with each other and their magnetic moments just add. 

\end{edXsolution}

\edXgitlink{\giturl}{Source TeX}
\end{edXproblem}

\end{edXvertical}

%%%%%%%%%%%%%%%%%%%%%%%%%%%%%%%%%%%%%%%%%%%%%%%%%%

\begin{edXvertical}

%%%%%%%%%%%%%%%%%%%%%%%%%%%%%%%%%%%%%%%%

\begin{edXproblem}{Custom response}

\section*{Example of custom response}  

This problem demonstrates the use of a custom python script used for
checking the answer.

\begin{edXscript}

def sumtest(expect,ans):
    try:
        (a1,a2) = map(float,ans)
        return (a1+a2)==10
    except Exception as err:
        return {'ok': False, 'msg': 'Sorry, cannot evaluate your input ' + str(ans)}

\end{edXscript}

Enter two numbers which add up to 10:

\edXabox{expect=""
  type="custom"
  answers="1","9"
  prompts="x = ","y = "
  cfn="sumtest"
  inline="1" }%

\edXgitlink{\giturl}{Source TeX}
\end{edXproblem}

\end{edXvertical}

%%%%%%%%%%%%%%%%%%%%%%%%%%%%%%%%%%%%%%%%%%%%%%%%%%

\begin{edXvertical}

%%%%%%%%%%%%%%%%%%%%%%%%%%%%%%%%%%%%%%%%

\begin{edXproblem}{Formula response}

This problem demonstrates the use of a the ``formula response'' problem.

Let $f(x) = a x^2 + bx + c$.  Assume the coefficients $a$, $b$, $c$
are such that $f$ has two distinct real roots. Give an equation for
$\lambda$, the larger of the two roots, such that $f(\lambda)=0$.

Use \texttt{^} for exponentiation, e.g. \texttt{x^2} denotes $x^2$.
Explicitly include multiplication using \texttt{*}, e.g. \texttt{x*y} is
$xy$.  Standard mathematical functions may be employed, e.g. \texttt{sin(x)},
\texttt{sqrt(x)}, etc.

% note that the sampling range must be carefully chosen such that
% the argument of the sqrt does not go negative, for the default random
% numerical sampling checker to work.

\edXinline{$\lambda =$ }
\edXabox{expect="(-b + sqrt(b^2-4*a*c))/(2*a)" type="formula"
  samples="a,b,c@1,16,1:3,20,3#50" size="60" tolerance='0.01' inline='1'
  math="1" feqin="1" }%

\edXgitlink{\giturl}{Source TeX}
\end{edXproblem}

\end{edXvertical}

\end{edXsequential}

%%%%%%%%%%%%%%%%%%%%%%%%%%%%%%%%%%%%%%%%%%%%%%%%%%%%%%%%%%%%%%%%%%%%%%%%%%%%%

\begin{edXsequential}{Video and text}

\begin{edXvertical}

\begin{edXtext}{Sample show-hide text section}

\section*{Sample show-hide text section}

Pieces of text (and images) can be put inside a ``showhide'' section, which the user can hide or show via a click.

\begin{edXshowhide}{ps4starkket}{Hints and instructions for entering expressions}

Note that rotations leave points on the axis of rotation unmoved.  For
a point $(\theta, \phi)$ specified in polar coordinates on the surface
of the \href{http://en.wikipedia.org/wiki/Bloch_sphere}{Bloch sphere},
the corresponding two-level quantum state is
\begin{eqnarray}
	|\psi\> = \cos\frac{\theta}{2} |a\> +
        e^{i\phi}\sin\frac{\theta}{2} |b\>
\,.
\end{eqnarray}

Enter the state using the vertical bar \texttt{|} and greater-than
symbol \texttt{>} to delineate a ``ket'' and enter $\omega_0$ as \texttt{omega_0} as usual.

Expressions like \texttt{(2*|a> + |b>)/sqrt(3)} are legal input
(remember to include \texttt{*} to denote multiplication, for
coefficients in front of kets).

Standard mathematical functions may be employed, e.g. \texttt{sin(x)},
\texttt{sqrt(x)}, \texttt{arccos(x)},  \texttt{arctan(x)}, etc.  

\end{edXshowhide}

\end{edXtext}

\end{edXvertical}

\begin{edXvertical}

\edXvideo{Introduction to 6.SFMx}{VfOMvuHcJi0}[url_name="week0_intro"]

\end{edXvertical}

\end{edXsequential}

%%%%%%%%%%%%%%%%%%%%%%%%%%%%%%%%%%%%%%%%%%%%%%%%%%%%%%%%%%%%%%%%%%%%%%%%%%%%%
\end{edXchapter}
